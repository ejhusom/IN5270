\documentclass[10pt, a4paper]{amsart}
%\usepackage[english,norsk]{babel}
\usepackage[english]{babel}
\usepackage[utf8]{inputenc}

\usepackage{geometry}   % Margins
\usepackage{graphicx}   % Images
\usepackage{float}      % Image floating
\usepackage{siunitx}    % SI-units
\usepackage{amsmath}
\newcommand{\bs}[1]{\boldsymbol{#1}}
\usepackage{algorithm}
\usepackage{verbatim}
\usepackage{url}
\usepackage{hyperref}
\usepackage{listings}
\usepackage{framed}
\numberwithin{figure}{section}
\numberwithin{table}{section}
\bibliographystyle{plain}

\usepackage{color}
%\usepackage{multicol}
%\setlength\columnsep{14pt}

\definecolor{codegreen}{RGB}{0, 146, 146}
\definecolor{codegray}{rgb}{0.4,0.4,0.4}
\definecolor{codeblue}{RGB}{0, 109, 219}
\definecolor{backcolour}{rgb}{0.9,0.9,0.9}

\lstdefinestyle{mystyle}{
    backgroundcolor=\color{backcolour},
    commentstyle=\color{codegreen},
    keywordstyle=\color{magenta},
    numberstyle=\tiny\color{codegray},
    stringstyle=\color{codeblue},
    basicstyle=\footnotesize,
    breakatwhitespace=false,
    showstringspaces=false,
    breaklines=true,
    captionpos=b,
    keepspaces=true,
    numbers=left,
    numbersep=5pt,
    showspaces=false,
    basicstyle=\footnotesize \ttfamily \color{black} \bfseries,
    xleftmargin=0.4cm,
    frame=tlbr, framesep=0.1cm, framerule=0pt,
    showtabs=false,
    tabsize=2
}

\lstset{style=mystyle}


\title[IN5270 Wave project]{IN5270 \\ \large
Wave project}
\author[Husom]{Erik Johannes B. L. G. Husom \\ \\ \today}


\begin{document}
\maketitle

%\begin{center}
%    Source code found in GitHub repository at \url{https://github.com/ejhusom/FYS-STK4155/}
%\end{center}

\tableofcontents



\section{Discretization of equations}

In this project we have the following 2D linear wave equation with damping:

\begin{align}
    \label{eq:wave}
\frac{\partial^2 u}{\partial t^2} + b \frac{\partial u}{\partial t} &=
\frac{\partial}{\partial x}\left( q(x,y) \frac{\partial u}{\partial x} \right)
    + \frac{\partial}{\partial y} \left( q(x,y) \frac{\partial u}{\partial
            y}\right) + f(x,y,t)
\end{align}
The boundary condition is
\begin{align}
\frac{\partial u}{\partial n} &= 0
\end{align}
and the inital conditions are

\begin{align}
u(x,y,0) &= I(x,y) \\
u_t(x,y,0) &= V(x,y)
\end{align}

To use this in our computer calculations, we need a discretized version. Since
we have a variable coefficient $q$, we write the inner derivatives (by using a
centered derivative) as:

\begin{align}
    \phi_x = q[x,y] \frac{\partial u}{\partial x}, \hspace{0.3cm}\phi_y =
    q[x,y] \frac{\partial u}{\partial y}
\end{align}

Then we get

\begin{align}
    \big[ \frac{\partial \phi_x}{\partial x}\big]_i^n &\approx \frac{\phi_{x,
    i+1/2} - \phi_{x,i-1/2}}{\Delta x} \\
    \big[ \frac{\partial \phi_y}{\partial y}\big]_j^n &\approx \frac{\phi_{y,
    j+1/2} - \phi_{y,j-1/2}}{\Delta y}
\end{align}

We then write

\begin{align}
    \phi_{x,i+1/2} & = q_{i+1/2,j} \big[\frac{\partial u}{\partial
    x}\big]_{i+1/2}^n \approx q_{i+1/2,j} \frac{u_{i+1,j} - u_{i,j}^n}{\Delta
    x}\\
    \phi_{x,i-1/2} & = q_{i-1/2,j} \big[\frac{\partial u}{\partial
    x}\big]_{i-1/2}^n \approx q_{i-1/2,j} \frac{u_{i,j} - u_{i-1,j}^n}{\Delta
    x}\\
    \phi_{y,j+1/2} & = q_{i,j++1/2} \big[\frac{\partial u}{\partial
    y}\big]_{j+1/2}^n \approx q_{i,j+1/2} \frac{u_{i,j+1} - u_{i,j}^n}{\Delta
    y}\\
    \phi_{y,j-1/2} & = q_{i,j-1/2} \big[\frac{\partial u}{\partial
    y}\big]_{j-1/2}^n \approx q_{i,j-1/2} \frac{u_{i,j} - u_{i,j-1}^n}{\Delta
    y}
\end{align}


To obtain $q$ at the half steps we use the arithmetic mean $q_{i+1/2} \approx
\frac{1}{2}(q_i + q_{i+1}$ and $q_{i-1/2} \approx \frac{1}{2}(q_i + q_{i-1}$.

This is then used to discretize equation \ref{eq:wave}:


\begin{align*}
\frac{u_{i,j}^{n+1}-2u_{i,j}^n+u_{i,j}^{n-1}}{\Delta t^2} + & b
\frac{u_{i,j}^{n+1}-u_{i,j}^{n-1}}{2\Delta t} = \frac{1}{\Delta x}
\left(q_{i+\frac{1}{2},j} \left( \frac{u_{i+1,j}^n - u_{i,j}^n}{\Delta x}\right)
    - q_{i-\frac{1}{2},j} \left( \frac{u_{i,j}^n - u_{i-1,j}^n}{\Delta x}\right)\right) \\
    & + \frac{1}{\Delta y} \left( q_{i,j+\frac{1}{2}} \left( \frac{u_{i,j+1}^n 
    - u_{i,j}^n}{\Delta y} \right) - q_{i,j-\frac{1}{2}} \left( \frac{u_{i,j}^n
    - u_{i,j-1}^n}{\Delta y}\right)\right) + f_{i,j}^n \\
\frac{u_{i,j}^{n+1}-2u_{i,j}^n+u_{i,j}^{n-1}}{\Delta t^2} &= - b
\frac{u_{i,j}^{n+1}-u_{i,j}^{n-1}}{2\Delta t} \\ & + \frac{1}{\Delta x^2} \left(
        \frac{1}{2}\left( q_{i+1,j} - q_{i,j} \right) \left(u_{i+1,j}^n 
        - u_{i,j}^n \right) - \frac{1}{2}  \left(q_{i,j}-q_{i-1,j} \right)
        \left( u_{i,j}^n - u_{i-1,j}^n \right)\right) \\
        &+ \frac{1}{\Delta y^2} \left(
        \frac{1}{2}\left( q_{i,j+1} - q_{i,j} \right) \left(u_{i,j+1}^n 
        - u_{i,j}^n \right) - \frac{1}{2} \left(q_{i,j}-q_{i,j-1} \right)
        \left( u_{i,j}^n - u_{i,j-1}^n \right)\right) + f_{i,j}^n \\
u_{i,j}^{n+1}-2u_{i,j}^n+u_{i,j}^{n-1} &=  \frac{-b \Delta t}{2} (u_{i,j}^{n+1}-u_{i,j}^{n-1}) \\
        &+\frac{\Delta t^2}{2 \Delta x^2} \left(
        \left( q_{i+1,j} - q_{i,j} \right) \left(u_{i+1,j}^n 
        - u_{i,j}^n \right) - \left(q_{i,j}-q_{i-1,j} \right)
        \left( u_{i,j}^n - u_{i-1,j}^n \right)\right) \\
        &+ \frac{\Delta t^2}{2 \Delta y^2} \left(
        \left( q_{i,j+1} - q_{i,j} \right) \left(u_{i,j+1}^n 
        - u_{i,j}^n \right) - \left(q_{i,j}-q_{i,j-1} \right)
        \left( u_{i,j}^n - u_{i,j-1}^n \right)\right) + \Delta t^2f_{i,j}^n \\
u_{i,j}^{n+1} + \frac{b \Delta t}{2}u_{i,j}^{n+1} &= 2u_{i,j}^n-u_{i,j}^{n-1}
+ \frac{b \Delta t}{2}u_{i,j}^{n-1} \\
 &+\frac{\Delta t^2}{2 \Delta x^2} \left(
        \left( q_{i+1,j} - q_{i,j} \right) \left(u_{i+1,j}^n 
        - u_{i,j}^n \right) - \left(q_{i,j}-q_{i-1,j} \right)
        \left( u_{i,j}^n - u_{i-1,j}^n \right)\right) \\
        &+ \frac{\Delta t^2}{2 \Delta y^2} \left(
        \left( q_{i,j+1} - q_{i,j} \right) \left(u_{i,j+1}^n 
        - u_{i,j}^n \right) - \left(q_{i,j}-q_{i,j-1} \right)
        \left( u_{i,j}^n - u_{i,j-1}^n \right)\right) + \Delta t^2f_{i,j}^n \\
\left(1+\frac{b \Delta t}{2}\right)u_{i,j}^{n+1} &= 2u_{i,j}^n - 
\left(1 - \frac{b \Delta t}{2}\right)u_{i,j}^{n-1} \\
&+\frac{\Delta t^2}{2 \Delta x^2} \left(
        \left( q_{i+1,j} - q_{i,j} \right) \left(u_{i+1,j}^n 
        - u_{i,j}^n \right) - \left(q_{i,j}-q_{i-1,j} \right)
        \left( u_{i,j}^n - u_{i-1,j}^n \right)\right) \\
        &+ \frac{\Delta t^2}{2 \Delta y^2} \left(
        \left( q_{i,j+1} - q_{i,j} \right) \left(u_{i,j+1}^n 
        - u_{i,j}^n \right) - \left(q_{i,j}-q_{i,j-1} \right)
        \left( u_{i,j}^n - u_{i,j-1}^n \right)\right) + \Delta t^2f_{i,j}^n \\
u_{i,j}^{n+1} &= \frac{1}{1+\frac{b \Delta t}{2}} \left( \right.
2u_{i,j}^n - 
\left(1 - \frac{b \Delta t}{2}\right)u_{i,j}^{n-1} \\
&+\frac{\Delta t^2}{2 \Delta x^2} \left(
        \left( q_{i+1,j} - q_{i,j} \right) \left(u_{i+1,j}^n 
        - u_{i,j}^n \right) - \left(q_{i,j}-q_{i-1,j} \right)
        \left( u_{i,j}^n - u_{i-1,j}^n \right)\right) \\
        &+ \frac{\Delta t^2}{2 \Delta y^2} \left(
        \left( q_{i,j+1} - q_{i,j} \right) \left(u_{i,j+1}^n 
        - u_{i,j}^n \right) - \left(q_{i,j}-q_{i,j-1} \right)
        \left( u_{i,j}^n - u_{i,j-1}^n \right)\right) + \Delta t^2f_{i,j}^n
        \left. \right)
\end{align*}

The modified scheme for the first step will find by using the initial
conditions:

\begin{align*}
u_t(x,y,0) &= V(x,y) \\
\frac{u_{i,j}^0 - u_{i,j}^{-1}}{\Delta t} &= V_{i,j} \\
u_{i,j}^{-1} &= u_{i,j}^0 - \Delta t V_{i,j}
\end{align*}



The modified scheme at the boundary points is found by using the boundary
conditions:


\begin{align*}
    \frac{\partial u}{\partial n} &= 0\\
\frac{u_{N_x + 1,j}^n - u_{N_x-1,j}^n}{2\Delta x} &= 0 \\
u_{N_x+1,j}^n &= u_{N_x-1,j}^n
\end{align*}


\section{Constant solution}

The file \texttt{test.py} contains test of a constant solution for both the
scalar and the vectorized version. With a constant solution $u(x,y,t) = c$, the derivatives
become zero, so if we set $f = 0$, our exact solution becomes the same as the
initial condition $I$, which has to be set to $I=c$. In my test case, I have
chosen $c=8$. The code also uses \texttt{nose.tools} to assert that the
difference between the exact and numerical solution is zero.


\section{Undampened waves}

For undampened waves, we have the exact solution

\begin{equation}
    u_e(x, y, t) = A \cos(k_x x) \cos (k_y y) \cos (\omega t),
\end{equation}

where

\begin{align}
    k_x &= \frac{m_x \pi}{L_x},\\
    k_y &= \frac{m_y \pi}{L_y}.
\end{align}

The variables $m_x$ and $m_y$ are arbitrary numbers, and $A$ is the
amplitude. In order to

\section{Manufactured solution}

I did not manage to get results that were consistent with the exact solutions.
I prioritized to finish the physical problem solving instead of finding the
bugs in this part of the project.

For the manufactured solution, I used \texttt{sympy} to obtain the following
expression:

\begin{align}
    f(x, y, t) &= A*kx**2*\cos(kx*x)*\cos(ky*y)*\cos(\omega*t)\\
    & + A*ky**2*\cos(kx*x)*\cos(ky*y)*\cos(\omega*t) \\
    &-A*\omega**2*\cos(kx*x)*\cos(ky*y)*\cos(\omega*t)
\end{align}

\section{Physical problem}

In this task I will investigate how two different kinds of bottom surfaces will
influence how the wave moves.

The first bottom shape, \texttt{B1}, is given by the formula:

\begin{equation}
    B = B0 + B_a \exp\big(-\big(\frac{x - B_{mx}}{B_s}\big)^2- \big(\frac{y -
    B_{my}}{b*B_s}\big)^2\big),
\end{equation}
where $b$ is a scaling parameter, and the other parameters affect the specific
shape of the bottom.

The other bottom shape I tested, \texttt{B2}, is given by:

\begin{equation}
    B = B_0 + B_a \cos\big(\pi \frac{x - B_{mx}}{2B_s}\big) \cos\big(\pi
    \frac{y - B_{my}}{2B_s}\big).
\end{equation}

Both of these bottom shapes are two of the shapes suggested in the project
description. In each case I set

$$B_0 = 0, B_a = 2.5, B_{mx} = B_{my} = 1, B_s = 0.4, b = 1$$

I chose to simulate the waves for a period of $T=2$, with a time step of
$0.1 h$, and $\Delta x = \Delta y = h$. I chose different values of $h$, which
are presented for each of the plots in this report.

To simplify the presentation of the results, I have chosen to produce gifs of
the wave motions, but I will also include some selected plots for different
runs in this report. Here is a list of the gifs, which are uploaded to the same
GitHub repository as this report (I was not able to produce gifs with $h <
0.1$, because the process got killed by my laptop):

\begin{itemize}
    \item \texttt{wave1h01.gif}, shape \texttt{B1}, $h=0.1$.
    \item \texttt{wave1h04.gif}, shape \texttt{B2}, $h=0.4$.
    \item \texttt{wave2h01.gif}, shape \texttt{B1}, $h=0.1$.
    \item \texttt{wave2h04.gif}, shape \texttt{B2}, $h=0.4$.
\end{itemize}


Figure \ref{fig:wave1h01} shows the \texttt{B1} shape with $h=0.1$, while
figure \ref{fig:wave1h04} shows the same shape with $h=0.4$. In figure
\ref{fig:wave2h01} and \ref{fig:wave2h04} we see bottom shape \texttt{B2} with
respectively $h=0.1$ and $h=0.4$. It is easier to inspect the results by
looking at gifs (which are placed in the same repository as this report), and
we can observe that the wave motion appears much smoother when we use a higher
resolution for the simulations. In figure \ref{fig:wave1h001} we can see the to
bottom shapes for an even higher resolution, with $h=0.01$.

\begin{figure}
       \includegraphics[width=\linewidth]{../src/wave1h01_1.png}
       \includegraphics[width=\linewidth]{../src/wave1h01_2.png}
       \caption{Bottom shape \texttt{B1}, $h=0.1$, for the first (left) 
       and last (right) timestep.}
       \label{fig:wave1h01}
\end{figure}

\begin{figure}
       \includegraphics[width=\linewidth]{../src/wave1h04_1.png}
       \includegraphics[width=\linewidth]{../src/wave1h04_2.png}
       \caption{Bottom shape \texttt{B1}, $h=0.4$, for the first (left) 
       and last (right) timestep.}
       \label{fig:wave1h04}
\end{figure}

\begin{figure}
       \includegraphics[width=\linewidth]{../src/wave2h01_1.png}
       \includegraphics[width=\linewidth]{../src/wave2h01_2.png}
       \caption{Bottom shape \texttt{B2}, $h=0.1$, for the first (left) 
       and last (right) timestep.}
       \label{fig:wave2h01}
\end{figure}

\begin{figure}
       \includegraphics[width=\linewidth]{../src/wave2h04_1.png}
       \includegraphics[width=\linewidth]{../src/wave2h04_2.png}
       \caption{Bottom shape \texttt{B2}, $h=0.4$, for the first (left) 
       and last (right) timestep.}
       \label{fig:wave2h04}
\end{figure}
\begin{figure}
       \includegraphics[width=\linewidth]{../src/wave1h001_1.png}
       \includegraphics[width=\linewidth]{../src/wave2h001_1.png}
       \caption{Bottom shape \texttt{B1} (left), and bottom shape \texttt{B2}
       (right), $h=0.01$.}
       \label{fig:wave1h001}
\end{figure}


\end{document}

